\documentclass[12pt]{article}
\usepackage[utf8]{inputenc}
\usepackage[spanish]{babel}
\usepackage[a4paper, margin=2.5cm]{geometry}
\usepackage{graphicx}
\usepackage{hyperref}
\usepackage{listings}
\usepackage{xcolor}
\usepackage{titlesec}
\usepackage{enumitem} 

\definecolor{lightgray}{gray}{0.95}
\lstset{
  backgroundcolor=\color{lightgray},
  basicstyle=\ttfamily\small,
  frame=single,
  breaklines=true
}

\titleformat{\section}{\normalfont\Large\bfseries}{\thesection.}{1em}{}

\title{Informe Técnico\\ \large EVA – Evaluador Virtual Académico}
\author{Darío López Falcón\\\\Facultad de Matemática y Computación (MATCOM)\\Universidad de La Habana}
\date{\today}

\begin{document}

\maketitle

\section*{Resumen}
Este proyecto tiene como objetivo automatizar la evaluación de respuestas académicas abiertas mediante modelos de lenguaje (LLMs), reduciendo la carga de trabajo docente y mejorando la objetividad.\\
\noindent\hspace*{1em}Se diseñó una aplicación web compuesta por un backend en Flask y un frontend en Next.js.\\
\noindent\hspace*{1em}La evaluación se realiza a través de prompts personalizados enviados a modelos de lenguaje de Fireworks.ai.\\
\noindent\hspace*{1em}El sistema permite a los profesores gestionar preguntas, ver resultados y obtener retroalimentación automática.\\
\noindent\hspace*{1em}A pesar de su efectividad en preguntas cerradas, presenta desafíos en preguntas abiertas que requieren razonamiento complejo.

\section*{Objetivos}
\textbf{Objetivo general:}
\begin{itemize}
  \item Automatizar la evaluación de respuestas escritas utilizando modelos de lenguaje.
\end{itemize}

\textbf{Objetivos específicos:}
\begin{itemize}
  \item Disminuir el tiempo de corrección de respuestas abiertas.
  \item Garantizar consistencia y objetividad en las calificaciones.
  \item Proporcionar retroalimentación automática a los estudiantes.
  \item Diseñar un sistema modular, extensible y fácil de desplegar.
\end{itemize}

\section{Definición del problema}
La corrección de respuestas abiertas suele ser lenta, subjetiva y difícil de estandarizar. Este proyecto busca una solución automatizada que mantenga coherencia en las calificaciones, permita escalabilidad y ofrezca retroalimentación inmediata al estudiante.

\section{Propuesta de solución}
Se desarrolló una aplicación web con dos módulos principales:
\begin{enumerate}
  \item \textbf{Backend:} construido con Flask, se encarga de la evaluación automática mediante un modelo LLM y gestiona los datos.
  \item \textbf{Frontend:} desarrollado con Next.js y TailwindCSS, permite a los docentes crear preguntas, filtrar resultados y visualizar evaluaciones.
\end{enumerate}

\section{Tecnologías utilizadas}
\subsection*{Frontend}
\begin{itemize}
  \item Next.js (React + SSR)
  \item TailwindCSS
  \item TypeScript
\end{itemize}

\subsection*{Backend}
\begin{itemize}
  \item Flask + SQLAlchemy + Flask-Migrate
  \item Python 3.12+
  \item API de Fireworks.ai
\end{itemize}

\section{Justificación de herramientas}
Las tecnologías fueron seleccionadas por su eficiencia, comunidad activa y facilidad de integración. Flask ofrece una solución ligera para APIs; Next.js permite crear interfaces rápidas y modernas; Fireworks.ai provee LLMs de alta calidad.

\section{Requisitos del sistema}
\textbf{Requisitos técnicos:}
\begin{itemize}
  \item Python 3.12+, Node.js, navegador moderno
  \item Conexión a internet y clave de API para Fireworks.ai
\end{itemize}

\textbf{Requisitos funcionales:}
\begin{itemize}
  \item Crear, editar y evaluar preguntas académicas
  \item Filtrar resultados, visualizar puntuaciones y retroalimentación
\end{itemize}

\section{Instalación y despliegue (Linux)}
\textbf{Backend:}
\begin{lstlisting}[language=bash]
git clone https://github.com/ArcanoxXx-01/EVA.git
cd EVA/backend
python3 -m venv env
source env/bin/activate
pip install -r requirements.txt
echo FIREWORKS_API_KEY=tu-api-key > .env
python run.py
\end{lstlisting}

\textbf{Frontend:}
\begin{lstlisting}[language=bash]
cd ../frontend
npm install
npm run dev
\end{lstlisting}

\section{Estructura del proyecto}

La estructura básica del proyecto se organiza de la siguiente manera:

\begin{itemize}
  \item \texttt{EVA/}
  \begin{itemize}
    \item \texttt{backend/}
    \begin{itemize}
      \item \texttt{app/} \quad (contiene \texttt{models}, \texttt{routes}, \texttt{services}, \texttt{llm})
      \item \texttt{run.py}
      \item \texttt{requirements.txt}
    \end{itemize}
    \item \texttt{frontend/}
    \begin{itemize}
      \item \texttt{app/}, \texttt{public/}, \texttt{src/}
      \item \texttt{package.json}, \texttt{tsconfig.json}, etc.
    \end{itemize}
  \end{itemize}
\end{itemize}

\section{Diseño y arquitectura}
La aplicación sigue una arquitectura desacoplada. El backend se comunica con el modelo LLM a través de una interfaz abstracta `BaseLLM', lo cual permite cambiar de proveedor fácilmente y mantener el sistema escalable y modular.

\section{Evaluación automática con Fireworks.ai}
El backend genera un prompt estructurado con:
\begin{itemize}
  \item Tema
  \item Tipo de pregunta
  \item Solución esperada
  \item Respuesta del estudiante
  \item Criterio de evaluación
\end{itemize}
El modelo devuelve una calificación numérica y una justificación textual.

\section{Pruebas y validación}
Las pruebas se centraron en:
\begin{itemize}
  \item Preguntas de opción múltiple y verdadero/falso, con alta precisión en la evaluación.
  \item Verificación del correcto armado de prompts y retorno esperado del modelo.
  \item Validación funcional del frontend (filtros, navegación, modo oscuro).
\end{itemize}
% No se realizó validación cuantitativa entre la evaluación automática y la humana (pendiente).

\section{Análisis del proyecto}
\subsection*{Fortalezas}
\begin{itemize}
  \item Arquitectura modular y desacoplada.
  \item Evaluación funcional de tipos de preguntas cerradas.
  \item Interfaz moderna e intuitiva.
\end{itemize}

\subsection*{Limitaciones}
\begin{itemize}
  \item No se persisten las evaluaciones generadas.
  \item Desempeño variable en preguntas abiertas.
  \item Ausencia de comparación sistemática con evaluadores humanos.
\end{itemize}

\section{Trabajos a futuro}

A partir del desarrollo actual del sistema, se identifican múltiples líneas de trabajo que permitirían ampliar su funcionalidad, robustez y utilidad en entornos reales:

\begin{itemize}
  \item \textbf{Persistencia de evaluaciones:} actualmente las respuestas generadas por el modelo no se almacenan. Guardar estos datos permitiría análisis estadísticos, validación posterior y trazabilidad de evaluaciones.
  
  \item \textbf{Historial por estudiante y problema:} agregar una funcionalidad que almacene los resultados de cada intento por estudiante ayudaría a monitorear la evolución del aprendizaje y detectar inconsistencias o mejoras.

  \item \textbf{Refinamiento del prompt:} mejorar el diseño del prompt utilizado con el modelo, empleando estrategias como few-shot learning, ejemplos concretos o ajuste dinámico según el tipo de pregunta.

  \item \textbf{Autenticación y roles de usuario:} implementar un sistema de inicio de sesión con permisos diferenciados (docente, administrador, etc.) para mejorar la seguridad y escalabilidad del sistema.

  \item \textbf{Soporte para múltiples modelos:} permitir el uso de distintos proveedores de LLMs (OpenAI, Cohere, Mistral...) usando la interfaz común ya existente (`BaseLLM`), de modo que el sistema sea más flexible y se puedan comparar diferentes motores de evaluación.

  \item \textbf{Visualización y exportación de resultados:} incluir reportes automáticos en PDF, Excel o gráficos de rendimiento por estudiante, problema o criterio de evaluación.

  \item \textbf{Feedback visual:} mostrar claramente el razonamiento generado por el modelo, indicando por qué una respuesta fue considerada correcta o incorrecta, lo que facilita la comprensión del estudiante.

  \item \textbf{Comparación con evaluaciones humanas:} realizar estudios de correlación entre evaluaciones hechas por el sistema y docentes reales, lo cual permitiría medir la precisión y confiabilidad del sistema en distintos contextos.

  \item \textbf{Uso de técnicas avanzadas de IA:} explorar el uso de \textit{fine-tuning} sobre modelos base o técnicas de \textit{retrieval-augmented generation} (RAG) para mejorar la calidad de las evaluaciones, especialmente en preguntas abiertas.

  \item \textbf{Sistema de colas y tareas asíncronas:} implementar una arquitectura con workers y procesamiento en segundo plano (por ejemplo, usando Celery o RQ) para manejar grandes volúmenes de evaluaciones sin afectar el rendimiento del servidor.
\end{itemize}

\section{Conclusiones}
El proyecto demuestra que es posible automatizar la evaluación académica con modelos LLM de forma efectiva en preguntas cerradas. La arquitectura desacoplada y el uso de una interfaz común para LLMs hacen que el sistema sea mantenible y extensible. Como trabajo futuro se propone mejorar la persistencia, refinar los prompts, y validar más rigurosamente la calidad de las respuestas del modelo.

\section*{Licencia}
Este proyecto está licenciado bajo la \textbf{MIT License}.

\end{document}
